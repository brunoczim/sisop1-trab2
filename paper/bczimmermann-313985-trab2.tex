\documentclass[a4paper]{article}

\usepackage[T1]{fontenc}
\usepackage[portuguese]{babel}
\usepackage{hyphenat}

\usepackage{geometry}
 \geometry{
 a4paper,
 left=20mm,
 right=20mm,
 top=30mm,
 bottom=20mm,
 }

\usepackage{fancyhdr}
\pagestyle{fancy}
\rhead{Bruno Corrêa Zimmermann -- 313985}
\title{Paginação de Memória, TLB, e Localidade -- Estudo de Caso}
\author{Bruno Corrêa Zimmermann}
\date{20 de Abril de 2022}
\setlength{\headheight}{13pt}

\usepackage{amssymb}
\usepackage{enumitem}
\usepackage{upquote}
\usepackage{amsmath}
\usepackage{fancyvrb}
\usepackage{tgcursor}
\usepackage{placeins}
\usepackage{fancyvrb,minted}
\usepackage{xcolor}
\usepackage{graphicx}
\usepackage{biblatex}
\usepackage{csquotes}
\addbibresource{trab2.bib}
\definecolor{LightGray}{gray}{0.9}
\renewcommand{\baselinestretch}{1.5}
\newenvironment{code}[1]{
\VerbatimEnvironment
\begin{minted}[frame=single,baselinestretch=1,fontsize=\small]{#1}}{
\end{minted}
}

\begin{document}

%\fontsize{12}{12}
%\selectfont

\maketitle

NOTA: os códigos se encontram no repositório
https://github.com/brunoczim/sisop1-trab2

\section{Introdução}

Memória virtual é um mecanismo de abstração da memória física que dá a impressão
a um processo de que o espaço de memória onde ele está pertence todo a ele. Uma
forma de implementar esse mecanismo, usada em processadores x86-64, em sistemas
como Linux, é paginação de memória. Nesse sistema, a memória é dividida em
páginas de tamanho fixo, e, no caso de Linux x86-64, o tamanho é 4KiB.

Nesse esquema, há páginas virtuais e páginas físicas. Páginas virtuais são
mapeadas para páginas físicas, mas processos enxergam apenas as páginas
virtuais. Para realizar a tradução entre endereços, é necessário manter tabelas
de páginas. As páginas virtuais podem não estar mapeadas ainda, e nesse caso,
ao tentar acessá-las, o processador gera um sinal chamado
``\textit{Page Fault}'', e então o sistema operacional mapeia a página virtual
para uma página física.

É possível que uma página seja armazenada em disco. Quando um processo tenta
acessar um endereço virtual, busca-se na tabela de páginas a entrada da página
virtual do endereço. Se não está mapeada (\textit{bit} de validade valendo $0$),
busca-se do disco a página e define-se o \textit{bit} de validade para $1$. Em
ambos os casos, ao final, obtém-se o número da página física, e dentro dela, o
enedereço físico é acessado.

Na tabela de páginas, há alguns \textit{bits} de controle para cada entrada.
O \textit{bit} mencionado acima (\textit{bit} de validade) indica se a entrada
na tabela é válida, ou seja, se há um mapeamento válido para a página virtual
associada à entrada. Há ainda o \textit{bit} ``\textit{dirty}'', que indica se a
página física associada àquela entrada na tabela está incosistente com o disco,
e no caso de substituição LRU (\textit{Least Recently Used}), um bit de
referência, indicando se essa entrada foi usada recentemente.

A tabela de páginas é finita, e portanto, quando uma nova página precisa ser
carregada, é possível que outra precise ser descarregada. Existem várias formas
de definir quais páginas serão descarregadas, e o LRU é um algoritmo que resolve
esse problema, substituindo as páginas que foram usadas pela última vez há mais
tempo. Outro algoritmo é o FIFO (\textit{First-in, First-out}), que substitui a
página carregada há mais tempo.

Além da tabela de páginas, existe um \textit{cache} de páginas, chamado TLB
(\textit{Translation-Lookaside Buffer}). O TLB se assemelha a tabela de
páginas convencional, mas é implementado em \textit{hardware}, então, se uma
página está no TLB, faz-se apenas um acesso à memória, em comparação aos
dois acesso pela tabela de página tradicional. Como é um \textit{cache}, ele não
armazena a estrutura toda das tabelas de páginas, mas apenas páginas
específicas, e portanto a TLB também guarda uma \textit{tag} para identificar a
página guardada em uma entrada.

Portanto, para aproveitar a performance do TLB, ou até mesmo da tabela de
páginas convencional, é importante que os programas busquem ter boa localidade.
Por exemplo, sequências de endereços que ``pulam'' páginas podem acabar fazendo
mau uso da TLB. Neste artigo, vamos estudar os efeitos de usar-se diferentes
estruturas de dados, com diferentes algoritmos, e com diferentes localidades.

\section{Localidade e Programas Estudados}

\nocite{*}
\printbibliography

\end{document}
